\section{HSM活动列表}

\subsection{接收活动列表}

\begin{lstlisting}[language={c++},numbers=left]
    /*  
    此API的目的是让copytool接收活动列表
    若返回值为0证明执行成功,返回负数则说明接收数据失败 
    */
    int llapi_hsm_copytool_recv(struct hsm_copytool_private *priv,
        struct hsm_action_list **hal, int *msgsize);
    /* 
    输入当前的copytool--->ct, 和一个活动表的指针--->hal,
    以及一个信息的bytes数量---> msgsize.
    按照bytes数量msgsize读取hal指向的活动表
    并将其传给priv指向的copytool
    */
\end{lstlisting}

\subsection{开始执行hsm活动表}

\begin{lstlisting}[language={c++},numbers=left]
    /*  
    此API的目的是让copytool开始执行活动列表
    此API必须在开始执行请求前,被copytool所调用
    若copytool仅仅想报一个错,则跳过此程序,跳转到llapi_hsm_action_end()
    若返回值为0证明执行成功,返回负数则说明执行失败 
    */
    int llapi_hsm_action_begin(struct hsm_copyaction_private **phcp,
        const struct hsm_copytool_private *ct,const struct hsm_action_item *hai,
        int restore_mdt_index, int restore_open_flags,bool is_error);
    /* 
    通过hcp的选择,该操作跳到llapi_hsm_action_progress() 和 llapi_hsm_action_end()
    ct指向copytool注册所需的hsm私有结构
    hai描述当前请求
    restore_mdt_index: 用于创建易变文件的MDT index,使用-1为默认值
    restore_open_flags: 易变文件创建模式
    is_error: 此次调用是否是仅仅为了返回一个错误
    */
\end{lstlisting}


\subsection{终止HSM活动表}
具体API如下:
\begin{lstlisting}[language={c++},numbers=left]
    /*  
    此API的目的是终止一个HSM活动进程
    该API只有在copytool终止处理所有请求后,才可被调用
    若返回值为0证明成功终止HSM活动进程,返回负数则说明终止进程失败 
    */
    int llapi_hsm_action_end(struct hsm_copyaction_private **phcp,
        const struct hsm_extent *he, int hp_flags, int errval);
    /*  
    he: 被复制数据最终的范围
    errval: 操作的状态码
    hp_flags: 终止状态的标志位
    */
\end{lstlisting}

\subsection{通知HSM进程}
具体API如下:
\begin{lstlisting}[language={c++},numbers=left]
    /* 
    此API的目的是在处理一个HSM行为时,通知一个进程
    若返回值为0证明成功,返回负数则说明失败 
    */
    int llapi_hsm_action_progress(struct hsm_copyaction_private *hcp,
        const struct hsm_extent *he, __u64 total, int hp_flags);
    /* 
    hcp 指向的是当前的复制行为
    he: 被复制数据的范围
    total: 所需全部复制数据
    hp_flags: HSM进程标志位
    */
\end{lstlisting}

\newpage
\subsection{获取被复制对象的文件标识符}
具体API如下:
\begin{lstlisting}[language={c++},numbers=left]
    /*
    此API的目的是获取被复制对象的文件标识符
    该文件标识符是通过传入的指针fid返回
    返回0如果操作成功,返回错误码如果此行为不是一个复制行为
    */
    int llapi_hsm_action_get_dfid(const struct hsm_copyaction_private *hcp, struct lu_fid *fid);
    /*
    hcp 为复制行为的指针
    fid为传入的指针,用于返回被复制对象的文件标识符
    */
\end{lstlisting}
该API会创建一个HSM活动项目:
\begin{lstlisting}[language={c++},numbers=left]
    /* 该结构是为了描述copytool项目行为的 */
    struct hsm_action_item {
        __u32      hai_len;     /* 结构的合理大小 */
        __u32      hai_action;  /* 用已知的大小构建的copytool行为 */
        struct lu_fid hai_fid;     /* 需要去操控的Lustre文件标识符 */
        struct lu_fid hai_dfid;    /* 用于获取数据的文件标识符*/
        struct hsm_extent hai_extent;  /* 需要操控的字节范围 */
        __u64      hai_cookie;  /* 协调器的行为cookie */
        __u64      hai_gid;     /* 组锁ID */
        char       hai_data[0]; /* 可变长度 */
    } __attribute__((packed));
\end{lstlisting}

\begin{lstlisting}[language={c++},numbers=left]
    // 若是以下两个条件不满足,则报错
    if (hcp->magic != CP_PRIV_MAGIC)
        return -EINVAL;
    if (hai->hai_action != HSMA_RESTORE && hai->hai_action != HSMA_ARCHIVE)
        return -EINVAL;
    // 上面创建的copytool项目行为的dfid, 就是待返回的dfid
    *fid = hai->hai_dfid;
\end{lstlisting}

\newpage
\subsection{获取文件描述符}
具体API如下:
\begin{lstlisting}[language={c++},numbers=left]
    /*
    此API的目的是获取文件描述符用于复制数据
    该文件描述符是通过传入的copyaction指针返回
    返回0如果操作成功,返回负的错误码如果操作失败
    */
    int llapi_hsm_action_get_fd(const struct hsm_copyaction_private *hcp);
\end{lstlisting}

该API会创建一个HSM活动项目:
\begin{lstlisting}[language={c++},numbers=left]
    /* 该结构是为了描述copytool项目行为的 */
    struct hsm_action_item {
        __u32      hai_len;     /* 结构的合理大小 */
        __u32      hai_action;  /* 用已知的大小构建的copytool行为 */
        struct lu_fid hai_fid;     /* 需要去操控的Lustre文件标识符 */
        struct lu_fid hai_dfid;    /* 用于获取数据的文件标识符*/
        struct hsm_extent hai_extent;  /* 需要操控的字节范围 */
        __u64      hai_cookie;  /* 协调器的行为cookie */
        __u64      hai_gid;     /* 组锁ID */
        char       hai_data[0]; /* 可变长度 */
    } __attribute__((packed));
\end{lstlisting}

\begin{lstlisting}[language={c++},numbers=left]
    // 如果magic不对,报错
	if (hcp->magic != CP_PRIV_MAGIC)
		return -EINVAL;
    // 如果活动为Archive或Restore并且fd<0,则报错
	if (hai->hai_action == HSMA_ARCHIVE) {
		fd = dup(hcp->source_fd);
		return fd < 0 ? -errno : fd;
	} else if (hai->hai_action == HSMA_RESTORE) {
		fd = dup(hcp->data_fd);
		return fd < 0 ? -errno : fd;
	} else {
		return -EINVAL;
	}
\end{lstlisting}

\subsection{引入已存在的HSM归档文件}
具体API如下:
\begin{lstlisting}[language={c++},numbers=left]
int llapi_hsm_import(
    const char *dst,    // dst指向Lustre目的地路径
    int archive,        // archive指向档案数
    const struct stat *st, // st包含文件所属关系,置换等
    // stripe:条目选项,当前默认忽视。
    unsigned long long stripe_size, 
    int stripe_offset,
    int stripe_count, 
    int stripe_pattern, 
    char *pool_name,
    struct lu_fid *newfid   // newfid: 输入新的Lustre文件的fid,作为输出
);
\end{lstlisting}
调用完此API后,调用者必须用获得的新文件标识符去获取文件
